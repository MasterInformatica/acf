\documentclass[10pt]{beamer}

\usetheme{m}

\usepackage{times}
\usepackage{tikz}
\usetikzlibrary{arrows,shapes}
\usepackage{siunitx}

\usepackage{amsmath}
\usepackage{fancyvrb}
\usepackage{xspace}

\usepackage{booktabs}
\usepackage[scale=2]{ccicons}

%\usepackage{pgfplots}
%\usepgfplotslibrary{dateplot}

\usepackage[latin1]{inputenc}
\usepackage[spanish]{babel}

\usepackage{listings}
\usepackage{pgfplots}


\title{L�gica Epist�mica}
\subtitle{}
\date{25 de enero de 2016}
\author{Luis Mar�a Costero Valero}
\institute{ACF - UCM}

%\titlegraphic{\hfill\includegraphics[scale=0.29]{images/logo.png}}

\definecolor{bgg}{HTML}{FBFBFB}
%\def\gcolor{bgg}    % while presenting
\def\gcolor{black} % while developing

\def\tikzpicdim{
  \draw[step=0.1cm, color=\gcolor] (0,0) grid (12,7);
  \draw[step=1cm, color=\gcolor] (0,0) grid (12,7);
}

\let\tikzpicdimlarge\tikzpicdim

\def\myurl{\hfil\penalty 100 \hfilneg \hbox}

\metroset{titleformat=regular}
\metroset{inner/sectiontitleformat=regular}
\metroset{outer/frametitleformat=regular}
\metroset{block=fill}


\newcommand\sectionDark[1]{{\metroset{background=dark} \section{#1} }}
\newcommand\sectionDarkSpecial[1]{{\metroset{background=dark} \section*{#1} }}

\begin{document}



\maketitle

\sectionDarkSpecial{Ejemplo}
\begin{frame}
  \frametitle{Ejemplo}
  
  \begin{columns}
    \begin{column}{0.55\textwidth}
      \onslide<1->{Suponemos que existen \alert{tres cartas} (1, 2 y 3):\\
        -- Alicia tiene una carta en la mano.\\
        -- Otra bocabajo en la mesa.\\
        -- La �ltima se devuelve al mont�n.\\
      }
      \vspace{0.5cm}
      \only<1,2>{\textbf{�Cu�les son los estados relevantes?}}
      \only<3,4,5>{\textbf{�Qu� informaci�n tiene Alicia?}}
    \end{column}
    \begin{column}{0.55\textwidth}
      \begin{tikzpicture}[scale=1.25, transform shape]
        \visible<2->{
          \tikzstyle{every node} = [circle, fill=gray!30, draw=gray!90] 
          \node [label=below left:\scriptsize$w1$] (w1) at (0.0, 4.0) {$\{1,2\}$};
          \node [label=below right:\scriptsize$w4$] (w4) at (1.7, 4.0) {$\{2,1\}$};
          \node [label=below left:\scriptsize$w2$] (w2) at (0.0, 2.0) {$\{1,3\}$};
          \node [label=below right:\scriptsize$w5$] (w5) at (1.7, 2.0) {$\{3,1\}$};
          \node [label=below left:\scriptsize$w3$] (w3) at (0.0, 0.0) {$\{2,3\}$};
          \node [label=below right:\scriptsize$w6$] (w6) at (1.7, 0.0) {$\{3,2\}$};
          \visible<4->{
            \draw [line width = 0.25mm, <->] (w1) -- (w2);
            \draw [line width = 0.25mm, <->] (w3) -- (w4);
            \draw [line width = 0.25mm, <->] (w5) -- (w6);
          }
          \visible<5->{
            \path [->] (w1) edge[loop left] ();
            \path [->] (w2) edge[loop left] ();
            \path [->] (w3) edge[loop left] ();
            \path [->] (w4) edge[loop right] ();
            \path [->] (w5) edge[loop right] ();
            \path [->] (w6) edge[loop right] ();
          }
        }
      \end{tikzpicture}
    \end{column}
  \end{columns}
\end{frame}



\end{document}
